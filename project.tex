%%% Local Variables: 
%%% mode: pdflatex
%%% TeX-master: t
%%% End: 

\documentclass[onecolumn,x11names,technote,twoside,a4paper,10pt,english]{IEEEtran}
\usepackage[english]{babel}
\usepackage[pdftex]{graphicx}
\usepackage{amssymb}
\usepackage{amsmath}
\usepackage{caption}
\usepackage{float}
\usepackage{tikz}
\usepackage{euler}                                %Nicer numbers


\begin{document}

\title{Project in Digital Communication Systems}
\author{Noam Lewis, Zvulun Avramov}

\maketitle

\section{Answers to Preparatory Questions}

We have selected modulation number 5, entailing digital input data, $R_b=800$, QAM, and $B_{ch}=800Hz$. Because $R_s=\frac{R_b}{log_2M}$, and the required bandwidth is $B_{min}=2R_s=800Hz$, we can calculate $M$:
\begin{eqnarray*}
  log_2M &=& \frac{R_b}{R_s} \\
  &=& \frac{2R_b}{B_{min}} \\
  &=& \frac{1600}{800} = 2
\end{eqnarray*}
Therefore, $M=4$ so we are using 4QAM. Choosing symmetric symbols yields $A_n \in \{\pm \sqrt{1/2} \pm j \sqrt{1/2} \}$ with $|A_n|=1$ and $\phi_n \in \{ \pi/4, 3\pi/4, 5\pi/4, 7\pi/4 \}$. 

\begin{enumerate}
\item The modulated message $s_M(t)$ is:
  \begin{eqnarray*}
    \label{eq:s_M}
    s_M(t) &=& Re\{s_d(t)\sqrt{2P_c}e^{j\omega_c t}\} \\
           &=& \sqrt{2P_c}\sum_n{|A_n| g(t-n T_s) cos(\omega_ct + \phi_n)} \\
           &=& \sqrt{2P_c}\sum_n{ |A_n| cos(\phi_n) g(t-n T_s) cos(\omega_ct) - |A_n| sin(\phi_n) g(t-n T_s) sin(\omega_ct)} \\
           &=& \sqrt{2P_c}\sum_n{ A_{n_i} g(t-n T_s) cos(\omega_ct) - A_{n_q} g(t-n T_s) sin(\omega_ct)}
  \end{eqnarray*}
  Where  $A_n=A_{n_i}+jA_{n_q}=|A_n|e^{j\phi_n}=e^{j\phi_n}$.
  In the case of $t \in K T_s$, where $K \in \mathbb{Z}$, we can use the fact that $\forall t \notin n T_s : g(t-n T_s)=0$, and $1$ otherwise to arrive at:
  \begin{eqnarray*}
    s_M(t \in K T_s) &=& \sqrt{2P_c} cos(\omega_c t + \phi_K) \\
                     &=& \sqrt{2P_c} \left[ cos(\phi_K)cos(\omega_c t) - sin(\phi_K)sin(\omega_c t) \right] \\
                     &=& \sqrt{2P_c} \left[ A_{K_i}cos(\omega_c t) - A_{K_q}sin(\omega_c t) \right] 
  \end{eqnarray*}
\item Using known results from DSB modulation, and because we have a full response filter, and assuming that the symbols have expectation $0$:
  \begin{eqnarray*}
    S_M(f) &=& \frac{P_c}{2}(S_d(f-f_c)+S_d(f+f_c)) \\
    G(f)   &=& T_s sinc(\pi f T_s)e^{-j\pi f T_s} \\
    |G(f)|^2 &=& T_s^2 sinc^2(\pi f T_s)        \\
    S_d(f) &=& \frac{\sigma_a^2}{T_s} |G(f)|^2 =  T_s\sigma_a^2 sinc^2(\pi f T_s) \\
    \sigma_a^2 &=& \sum_{k=1}^M{p_k A_k^2} = \frac{1}{4} 4 = 1
  \end{eqnarray*}
  Substituting $S_d$ into $S_m$ we get:
  \begin{equation}
    \label{eq:S_M(f)}
    S_M(f) = \frac{P_c}{2R_s}\left[ sinc^2(\pi T_s(f-f_c)) + sinc^2(\pi T_s(f+f_c)) \right]
  \end{equation}
  The graph of $S_M(f)$ with example parameters is show in Figure \ref{fig:S_M(f)}.
  \begin{figure}[H]
    \centering
    \input{q2.latex}
    \caption{$S_M(f)$ for $R_s=T_s^{-1}=10,f_c=100$}
    \label{fig:S_M(f)}
  \end{figure}
  When there is no noise in the channel, and with ideal synchronization the demodulated signal is:
    \begin{eqnarray*}
      x_{d_i}(t) &=& x(t)A_c cos(\omega_c t) \\
                &=& A_c\sqrt{2P_r} cos(\omega_c t + \phi_K) cos(\omega_c t) \\
                &=& A_c\sqrt{\frac{P_r}{2}} \left[ cos(\phi_k) + cos(2\omega_c t + \phi) \right] \\
      x_{d_q}(t) &=& -x(t)A_c sin(\omega_c t) \\
                &=& A_c\sqrt{\frac{P_r}{2}} \left[ sin(\phi_k) + sin(2\omega_c t + \phi) \right] 
    \end{eqnarray*}
  If $x_{d_i}(t)$ for example passes through the appropriate matched filter, the output is:
  \begin{eqnarray*}
    q_{k_i} &=& \int_{kT_s}^{(k+1)T_s}{ x_{d_i}(t) g(t-kT_s) dt} \\
           &=&  \int_{kT_s}^{(k+1)T_s}{ A_c\sqrt{\frac{P_r}{2}} \left[ cos(\phi_k) + cos(2\omega_c t + \phi) \right]  dt} \\
           &=&  A_c\sqrt{\frac{P_r}{2}} T_s cos(\phi_k) \int_{kT_s}^{(k+1)T_s}{ cos(2\omega_c t + \phi)  dt} 
  \end{eqnarray*}
  If $R_s \to m (2f_c)$ where $m \in \mathbb{N}$, the integral goes to zero, and we are left with:
  \begin{equation*}
    q_{k_i} = A_c\sqrt{\frac{P_r}{2}} T_s cos(\phi_k)
  \end{equation*}
  The conclusion is that to restore the amplitude of $q_{k_i}$ to the original (transmitted) value, we must multiply by $2 R_s$. To restore the value that entered the transmit filter, we must also divide by $A_c=\sqrt{2P_c}$. Equivalently to the two operations, we can simply multiply $q_{k_i}$ by the factor $R_s / \sqrt{P_c} = \frac{2}{A_c T_s}$.

\item For a symbol dictionary of size $M$: $R_s=\frac{R_b}{log_2M}$ and $K_b=log_2M$.

\item In Gray coding, every two consequetive numbers differ by exactly one bit. The encoding reduces BER by assigning similar codes to closer symbols. The probability of receiving a symbol in close vicinity to the transmitted one is higher, so errors with small number of bit flips are more probable. In our case (where $M=4$) we can assign:

  \begin{center}
    \begin{table}
      \centering
      \begin{tabular}[h!]{| l | l | l |}
      \hline
      Code & Symbol & \\ \hline
      $00$ & $+\sqrt{1/2}+j\sqrt{1/2}$ & $A_1$  \\ \hline
      $01$ & $-\sqrt{1/2}+j\sqrt{1/2}$ & $A_2$  \\ \hline
      $11$ & $-\sqrt{1/2}-j\sqrt{1/2}$ & $A_3$  \\ \hline
      $10$ & $+\sqrt{1/2}-j\sqrt{1/2}$ & $A_4$  \\
      \hline
      \end{tabular}
      \caption{Assignment of bit patterns to symbols}
      \label{tab:bitsymbols}
    \end{table}
  \end{center}

\item The symbol constellation is shown in Figure \ref{fig:4qam-const}. 
  \begin{figure}[H]
    \begin{center}
      \begin{tikzpicture}[scale=3]
        % Circles 
        \foreach \r in { 1}
        \draw[Azure4, thin] (0,0) circle (\r);
        % 45� Rays
        \foreach \a in {0, 45,...,359}
        \draw[Azure4] (\a:1) -- (\a:1.5);
        % Radius labels (background filled white)
        \draw (0.707,0) node[inner sep=1pt,below=3pt,rectangle,fill=white] {$\frac{1}{\sqrt{2}}$};
        \draw (0,0.707) node[inner sep=1pt,left=3pt,rectangle,fill=white] {$j\frac{1}{\sqrt{2}}$};
        \draw (-0.707,0) node[inner sep=1pt,below=3pt,rectangle,fill=white] {$-\frac{1}{\sqrt{2}}$};
        \draw (0,-0.707) node[inner sep=1pt,left=3pt,rectangle,fill=white] {$-j\frac{1}{\sqrt{2}}$};
        % Main rays
        \foreach \a in {0, 90,...,359}
        \draw[very thick] (0, 0) -- (\a:1.5);
        % Angle labels  
        \foreach \a in {0, 45,...,359}
        \draw (\a: 1.6) node {$\a^\circ$};
        % Central point
        \draw[fill=red] (0.707,0.707) circle(0.4mm) node[inner sep=1pt,below=4pt,rectangle] {$00$};
        \draw[fill=red] (-0.707,0.707) circle(0.4mm)  node[inner sep=1pt,below=4pt,rectangle] {$01$};
        \draw[fill=red] (-0.707,-0.707) circle(0.4mm)  node[inner sep=1pt,below=4pt,rectangle] {$11$};
        \draw[fill=red] (0.707,-0.707) circle(0.4mm)  node[inner sep=1pt,below=4pt,rectangle] {$10$};
      \end{tikzpicture}
    \end{center}
    \caption{Constellation of symbols. Each node is labeled with the assigned (gray coded) bit pattern.}
    \label{fig:4qam-const}
  \end{figure}

\item The Euclidean distance between adjacent symbols is $\frac{2}{\sqrt{2}}=\sqrt{2}$. In terms of phase drift, the distance is $\pi/4$.

\item Effects of channel on the communication system:
  \begin{enumerate}
  \item AWGN - The noise appears as an additive element in the symbols entering the decision device. If $a_k$ is the portion originating from the source signal, and $z_k$ is the effect of the noise, then $q_k = a_k + z_k$. For a time period where $t \in k T_s$ we have:
    \begin{eqnarray*}
      x(t) &=& s_r(t) + n_r(t) \\
      s_r(t) &=& k_{ch}s_M(t) = \sqrt{2P_r} \left[ cos(\phi_K)cos(\omega_c t) - sin(\phi_K)sin(\omega_c t) \right]  \\
      n_r(t) &=& n_{r_i}(t)cos(\omega_c t) - n_{r_q}(t) sin(\omega_c t) \\
    \end{eqnarray*}
    The noise component affects $\text{SNR}_d$. The relevant parameter is the noise power (variance) $\sigma_z^2$.

  \item Limited channel bandwidth - ISI (Inter Symbol Intereference): In this case the received symbol $q_k$ has an additional component, the ISI: $q_k = p_0A_K + \sum_{n \neq k} {A_n p_{k-n}} + z_k$. This effect can be countered by, for example, an optimal sequential detector.

  \item Phase and Frequency shift - after demodulation (separately for the two bases), we get:
    \begin{eqnarray*}
      x_{d_i}(t) &=& x(t)A_c cos(\omega_c t) \\
                &=& A_c\sqrt{2P_r} cos(\omega_r t + \phi_K + \Psi) cos(\omega_c t) + A_c cos(\omega_c t) n_r(t)\\
                &=& A_c\sqrt{\frac{P_r}{2}} cos(\phi_k + \Psi + \Delta \omega t) 
                    + \underbrace{ \frac{A_c}{2} \left( n_{r_i}(t) + n_{r_q}(t) \right) }_{\mbox{additive noise component}} 
                    + \text{ Higher frequency components} \\
      x_{d_q}(t) &=& -x(t)A_c sin(\omega_c t) = A_c\sqrt{\frac{P_r}{2}} sin(\phi_k + \Delta \omega t + \Psi) + \text{ ... }
    \end{eqnarray*}
    Where $\Delta \omega = \omega_r - \omega_c$ is the frequency shift, and $\Psi$ is the phase shift. Since we are using 4-QAM, if the phase shift nears $\pi/8$ the symbols may be confused, so a phase synchronizer is required.
    
  \end{enumerate}

  \item Definitions of $\text{SNR}_{bit}^{(DD)}$ and $\text{SNR}_{sym}^{(DD)}$: 
    
\item $erfc(\lambda) = 2Q( \sqrt{2}\lambda )$.

\item Because we are using code gray, the following relation holds whenever the SNR is high enough to make transitions to far symbols unlikely:
    \begin{equation*}
      P_{err,bit}^{(DD)} \approx \frac{P_{err,sym}^{(DD)}}{log_2{M}} = \frac{P_{err,sym}^{(DD)}}{2}
    \end{equation*}
    The probability for symbol error is defined by:
    \begin{equation*}
      P_{err,sym}^{(DD)} = \sum_{m=1}^M {p_m P(\text{error} | A_m)} = \sum_{m=1}^M {p_m \left[ 1 - \prod_{n=1,n \neq m}^M \left( 1 - P(A_n | A_m) \right) \right]} 
    \end{equation*}
    Where, if $d_{m,n}$ is the Euclidean distance between symbols $A_m$ and $A_n$:
    \begin{equation*}
      P(A_n | A_m) = Q\left( \frac{d_{m,n}}{2 \sigma_z}\right)
    \end{equation*}

\end{enumerate}

\section{Simulation}

\subsection{Data sets}
\begin{enumerate}
\item Small data set - one of each bit pattern: $\{00, 01, 11, 10\}$.
\item Large (random) data set - a random stream of $5^3 log_2M = 10^3$ bits. The data set was created by applying the function $f(x) = \text{\bf{ if }} x>0.5 \text{\bf{ then }} 1 \text{\bf{ else }} 0$ to each element of a vector constructed with the Matlab 'rand' function.
\end{enumerate}

\subsection{Transmitter}

The symbol/bit patterns appear in Table \ref{tab:bitsymbols}.
\begin{enumerate}
\item $P_c = 10W$, therefore $A_c = \sqrt{20}$.
\item $f_c = 20kHz$ so $w_c = 4\pi \cdot 10^4$.
\item Sampling frequency (for simulation of the continuous signal) is $f_s = 2f_c = 40khz$.
\end{enumerate}

  \begin{figure}[h!]
    \centering
    \includegraphics[height=4in]{modulated_small_dataset.png}
    \caption{$s_M(t)$ for small data set}
    \label{fig:s_M(t)}
  \end{figure}


\end{document}

