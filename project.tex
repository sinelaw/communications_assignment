%%% Local Variables: 
%%% mode: latex
%%% TeX-master: t
%%% End: 

\documentclass[onecolumn,x11names,technote,twoside,a4paper,10pt,english]{IEEEtran}
\usepackage[english]{babel}
\usepackage[pdftex]{graphicx}
\usepackage{amssymb}
\usepackage{amsmath}
\usepackage{caption}

\usepackage{tikz}
\usepackage{euler}                                %Nicer numbers


\begin{document}

\title{Project in Digital Communication Systems}
\author{Noam Lewis}

\maketitle

\section{Answers to Preparatory Questions}

We have selected modulation number 5, entailing digital input data, $R_b=800$, QAM, and $B_{ch}=800Hz$. Because $R_s=\frac{R_b}{log_2M}$, and the required bandwidth is $B_{min}=2R_s=800Hz$, we can calculate $M$:
\begin{eqnarray*}
  log_2M &=& \frac{R_b}{R_s} \\
         &=& \frac{2R_b}{B_{min}} \\
         &=& \frac{1600}{800} = 2
\end{eqnarray*}
Therefore, $M=4$ so we are using 4QAM. Choosing symmetric symbols yields $A_n \in \{\pm \sqrt{1/2} \pm j \sqrt{1/2} \}$ with $|A_n|=1$ and $\phi_n \in \{ \pi/4, 3\pi/4, 5\pi/4, 7\pi/4 \}$. 

\begin{enumerate}
\item The modulated message $s_M(t)$ is:
  \begin{eqnarray}
    \label{eq:s_M}
    s_M(t) &=& Re\{s_d(t)\sqrt{2P_c}e^{j\omega_c t}\} \\
    &=& \sqrt{2P_c}\sum_n{|A_n| g(t-n T_s) cos(\omega_ct + \phi_n)}
  \end{eqnarray}
  In the case of $t \in K T_s$, where $K \in \mathbb{Z}$, we can use the fact that $\forall t \notin n T_s : g(t-n T_s)=0$, and $1$ otherwise to arrive at:
  \begin{equation}
    s_M(K T_s) = \sqrt{2P_c}cos(\omega_c K T_s - \phi_K)
  \end{equation}
  Where  $A_n=|A_n|e^{j\phi_n}=e^{j\phi_n}$.
\item Using known results from DSB modulation, and because we have a full response filter, and assuming that the symbols have expectation $0$:
  \begin{eqnarray}
    S_M(f) &=& \frac{P_c}{2}(S_d(f-f_c)+S_d(f+f_c)) \\
    S_d(f) &=& T_s sinc(\pi f T_s)e^{-j\pi f T_s}
  \end{eqnarray}
  Substituting $S_d$ into $S_m$ we get:
  \begin{equation}
    \label{eq:S_M(f)}
    S_M(f) = \frac{P_c}{2R_s}\left[ sinc(\pi T_s(f-f_c)) e^{-j\pi(f-f_c)T_s} 
                           + sinc (\pi T_s(f+f_c))e^{-j\pi(f+f_c)T_s} \right]
  \end{equation}
  \begin{figure}[h]
    \label{fig:S_M(f)}
    \centering
    \input{q2.latex}
    \caption{$S_M(f)$ for $R_s=T_s^{-1}=10,f_c=100$}
  \end{figure}

\item For a symbol dictionary of size $M$: $R_s=\frac{R_b}{log_2M}$ and $K_b=log_2M$.

\item In Gray coding, every two consequetive numbers differ by exactly one bit. The encoding reduces BER by assigning similar codes to closer symbols. The probability of receiving a symbol in close vicinity to the transmitted one is higher, so errors with small number of bit flips are more probable. In our case (where $M=4$) we can assign:

\begin{center}
  \begin{tabular}{| l | l | }
    \hline
    Code & Symbol \\ \hline
    $00$ & $+\sqrt{1/2}+j\sqrt{1/2}$   \\ \hline
    $01$ & $-\sqrt{1/2}+j\sqrt{1/2}$  \\ \hline
    $11$ & $-\sqrt{1/2}-j\sqrt{1/2}$  \\ \hline
    $10$ & $+\sqrt{1/2}-j\sqrt{1/2}$   \\
    \hline
  \end{tabular}
\end{center}

\item The symbol constellation is shown in Figure \ref{fig:4qam-const}. 
\begin{figure}[h]
\begin{center}
  \begin{tikzpicture}[scale=3]
    %Circles 
    \foreach \r in { 1}
      \draw[Azure4, thin] (0,0) circle (\r);
    %45� Rays
    \foreach \a in {0, 45,...,359}
      \draw[Azure4] (\a:1) -- (\a:1.5);
    %Radius labels (background filled white)
    \draw (0.707,0) node[inner sep=1pt,below=3pt,rectangle,fill=white] {$\frac{1}{\sqrt{2}}$};
    \draw (0,0.707) node[inner sep=1pt,left=3pt,rectangle,fill=white] {$j\frac{1}{\sqrt{2}}$};
    \draw (-0.707,0) node[inner sep=1pt,below=3pt,rectangle,fill=white] {$-\frac{1}{\sqrt{2}}$};
    \draw (0,-0.707) node[inner sep=1pt,left=3pt,rectangle,fill=white] {$-j\frac{1}{\sqrt{2}}$};
    %Main rays
    \foreach \a in {0, 90,...,359}
      \draw[very thick] (0, 0) -- (\a:1.5);
    %Angle labels  
    \foreach \a in {0, 45,...,359}
      \draw (\a: 1.6) node {$\a^\circ$};
    %Central point
    \draw[fill=red] (0.707,0.707) circle(0.4mm) node[inner sep=1pt,below=4pt,rectangle] {$00$};
    \draw[fill=red] (-0.707,0.707) circle(0.4mm)  node[inner sep=1pt,below=4pt,rectangle] {$01$};
    \draw[fill=red] (-0.707,-0.707) circle(0.4mm)  node[inner sep=1pt,below=4pt,rectangle] {$11$};
    \draw[fill=red] (0.707,-0.707) circle(0.4mm)  node[inner sep=1pt,below=4pt,rectangle] {$10$};
  \end{tikzpicture}
\end{center}
    \caption{Constellation of symbols. Each node is labeled with the assigned (gray coded) bit pattern.}
\label{fig:4qam-const}
\end{figure}

\item The distance between adjacent symbols is $\frac{2}{\sqrt{2}}=\sqrt{2}$.

\end{enumerate}


\end{document}

